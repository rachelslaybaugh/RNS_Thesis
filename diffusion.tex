% Prelim, Appendix A
% by Rachel Slaybaugh

\chapter{Diffusion Equation}
\label{sec:AppendixA}
The diffusion approximation is a widely used simplification that reduces the computational complexity of the transport equation. The approximation is that the angular dependence of the flux is unimportant, so the direction component of the transport equation can be discarded. Physically this means that neutrons move against their concentration gradient like just heat diffuses through a conductor. The information in this section is derived from Duderstadt and Hamilton's \emph{Nuclear Reactor Analysis} \cite{Duderstadt1976} and neglects fission for simplicity. 

The first step in applying this approximation is to integrate the angular dependence out of Equation \eqref{eq:neutron transport}, resulting in the neutron continuity equation:
%
\begin{equation}
  \nabla \cdot J(\vec{r},E) + \Macro(\vec{r},E)\phi(\vec{r},E) = \int \:dE' \:\Macro_{s}(\vec{r}, E' \to E)\phi(\vec{r},E') + Q_{ex}(\vec{r},E) \:,
  \label{eq:continuity} 
\end{equation}
%
where the following definitions have been used:\\
\indent $J(\vec{r},E) = \int d\vOmega \:\vOmega \psi(\vec{r}, \vOmega, E)$ is the neutron current \\
\indent  $\phi(\vec{r},E) = \int d\vOmega \:\psi(\vec{r}, \vOmega, E)$ is the scalar flux, and \\
\indent $Q_{ex} (\vec{r},E)= \int d\vOmega \:q_{ex}(\vec{r}, \vOmega, E)$ is the external source.

Unfortunately, this simplifying approximation added another unknown, $J$, which leaves one equation with two unknowns. In an attempt to eliminate one of these unknowns, Equation \eqref{eq:continuity} is multiplied by $\hat{\Omega}$ and integrated over angle again to obtain the first angular moment:
%
\begin{align}
  \nabla \cdot \int  d\vOmega \:\vOmega \vOmega \psi(\vec{r}, \vOmega, E) &+ \Macro(\vec{r},E) J(\vec{r},E)= \nonumber \\
  &\int dE' \:\Macro_{s1}(\vec{r}, E' \to E)J(\vec{r},E') + \int d\vOmega \int d\vOmega \:\vOmega q_{ex} \:,
  \label{eq:current1}
\end{align}
%
where $\Macro_{s1}  = \int d\vOmega \:\vOmega \Macro_{s}$. The first angular moment form of the equation cannot be solved either because the streaming (first) term is still unknown. 

To make Equation \eqref{eq:current1} solvable, the original assumption is modified to assert that the angular flux is weakly, in fact linearly, dependent on angle rather than independent of angle. To implement this assumption the angular flux is expanded in angle and only the first two terms are retained:  
%
\begin{equation}
  \psi(\vec{r}, \vOmega, E) \cong \frac{1}{4 \pi} \phi(\vec{r}, E) + \frac{3}{4 \pi}J(\vec{r}, E) \cdot \vOmega \:.
  \label{eq:angExpand} 
\end{equation}
The truncated angular flux is then inserted into the streaming term in Equation \eqref{eq:current1}, giving 
%
\begin{equation}
  \nabla \cdot \int d \vOmega \:\vOmega \vOmega \psi(\vec{r}, \vOmega, E)  \cong \frac{1}{3} \nabla \phi(\vec{r}, E) \:. 
  \label{eq:firstTerm}
\end{equation}

Next, the scattering source term is simplified in angle and energy. To address the angular dependence define $\bar{\mu}_{0}$ as the average cosine of the scattering angle, which, temporarily suppressing energy, gives $\Macro_{s1} = \bar{\mu}_{0}\Macro_{s}$. For elastic scattering from stationary nuclei when s-wave scattering is present in the center of mass frame, $\bar{\mu_{0}} = \frac{2}{3A}$ where $A$ is atomic mass number. A common procedure to simplify the energy dependence is to neglect the anisotropic contribution to energy transfer in a scattering collision. Mathematically this means $\Macro_{s1}(E' \to E) = \Macro_{s1}(E) \delta(E' = E)$, giving $\int dE' \:\Macro_{s1}(\vec{r}, E' \to E)J(\vec{r},E') = \bar{\mu_{0}}\Macro_{s}(\vec{r},E)J(\vec{r},E)$. Finally, it is assumed that the external source is isotropic, $\int  d\vOmega \int d\vOmega \:\vOmega q_{ex} = 0$.

If these approximations are all included and Equation \eqref{eq:current1} is solved for $J$, the result is Fick's Law:
%
\begin{equation}
  J(\vec{r},E) \cong -\frac{1}{3(\Macro(\vec{r},E) - \bar{\mu_{0}}\Macro_{s}(\vec{r},E))} \nabla \phi (\vec{r},E) = -D(\vec{r},E) \nabla \phi (\vec{r},E) \:.
\end{equation}
%
Fick's Law can be introduced back into Equation \eqref{eq:continuity} to obtain the diffusion equation:
%
\begin{equation}
  -\nabla \cdot D(\vec{r},E) \nabla \phi(\vec{r},E) + \Macro(\vec{r},E)\phi(\vec{r},E) = \int dE' \:\Macro_{s}(\vec{r}, E' \to E)\phi(\vec{r},E') + q_{ex}(\vec{r},E) \:.
  \label{eq:diffusion}
\end{equation}

This equation now includes several assumptions which are valid when the solution is not near a void, boundary, source, or strong absorber. While these requirements can be quite restrictive, the diffusion equation has been used frequently for analysis of nuclear systems throughout the history of the nuclear industry.

Some terms from this section that are used when describing DSA in Chapter \ref{sec:Chp4} are:
\begin{align}
  \Macro_{tr} &= \Macro(\vec{r},E) - \bar{\mu_{0}}\Macro_{s}(\vec{r},E) \qquad \text{is the transport cross section, and}\\
  \ve{C} &= -\nabla \cdot \frac{1}{3\Macro_{tr}}\nabla+ \Macro(\vec{r},E) \qquad \text{is the diffusion operator.}
\end{align}

