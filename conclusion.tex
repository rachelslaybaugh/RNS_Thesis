% Prelim, Chapter 5
% by Rachel Slaybaugh

\chapter{Conclusions}
\label{sec:Chp5}
%This chapter provides a brief summary of the motivation for this research, the work that has been completed, and revisits the work that will be done. 

%To enhance and improve the design of nuclear systems like reactors, high-fidelity neutron fluxes are required. Denovo is a 3-D, \Sn code that has the potential to calculate such fluxes. It has been shown that Denovo scales well and can compute accurate solutions, making it a very promising tool for advanced nuclear computation. 

%However, there is still a need for improved methods. As demonstrated in Chapter \ref{sec:Chp1}, Denovo can only scale to $O(1000)$ cores. Further, some calculations still require many iterations and take a long time to converge. The work presented is intended to allow Denovo to be more effective for ``grand challenge'' calculations by developing solutions for overcoming these problems. 

%\section{Completed Work: Upscattering and Energy Decomposition}
%The transport equation can be formulated as either a fixed source problem or as an eigenvalue problem. When upscattering cross sections are needed, the common solution algorithm is Gauss Seidel. Gauss Seidel can be very slow to converge for problems that are highly scattering, and it is an implicit method that couples all upscattering energy groups such that they must be solved serially. 

%In this work, a block Krylov method was added for solving the upscattering groups in place of Gauss Seidel. Instead of sequentially solving each group with some inner iteration method and then using GS for outer iterations, all groups can be solved at once with a Krylov solver and the inner-outer iteration structure is removed. For many problems a Krylov method will converge more quickly than GS. The author did not find any other code that applied Krylov solvers over a block of groups without also needing inner iterations. 

%In addition to providing faster convergence, the block Krylov solver allows energy groups to be decoupled because the group equations are formulated to be explicit. This group decoupling allows the upscattering block to be broken into energy sets and each set can be solved with a Krylov solver by a different processor. The author is unaware of any other \Sn transport code that is parallelized in the energy dimension. This addition expands the number of processors that can be used efficiently by Denovo. 

%A few test problems were presented that show this new method works, that the Krylov solver is faster than the GS solver, and that the energy decomposition scales well. Denovo was used successfully on hundreds of thousands of processors because of the energy decomposition.

%\section{Proposed Work: Eigenvalue Acceleration}
%Eigenvalue calculations, whether or not they have upscattering, will be improved as well. Most \Sn codes, including Denovo, use power iteration or inverse iteration to solve the $k$-eigenvalue problem. Power iteration in particular can converge very slowly for problems with high dominance ratios. 

%Rayleigh quotient iteration will be implemented as a new choice to solve eigenvalue calculations. RQI is an inverse iteration method that creates an optimal shift and has good convergence properties. This shift will be applied to the scattering matrix to make it look like every group has upscattering. The shift will allow the block Krylov solver to be applied easily to eigenvalue problems, and the energy groups can be decoupled so eigenvalue calculations can be parallelized in this area of phase space as well. 

%The new eigenvalue solver should give faster convergence and enable parallelization in energy. RQI has never been applied to the 3-D neutron transport equation, and the scattering matrix has never been shifted to enable energy decomposition.

%The RQI+MG Krylov solver will be tested on a both toy and real problems to verify it is correct and to see how it works for problems of interest. It will be tested on problems that have high dominance ratios to see the impact of using RQI instead of PI. Scaling studies will also be done to investigate the energy decomposition provided by the block Krylov solver.

%\section{Proposed Work: Preconditioners}
%Another potential way to reduce the time to solution for a given calculation is by reducing the number of iterations needed for convergence through preconditioning. If the savings from a lower iteration count is more than cost of the extra work done in making and applying the preconditioner, the result is a faster code. 

%Two classes of preconditioners are proposed. The first is a multilevel energy preconditioner. Iterative methods reduce oscillatory error effectively, but not smooth error. If the transport equation is restricted to coarser energy grids, the error modes that appeared smooth on the fine grid become oscillatory. An iterative scheme can be applied on the coarse grids to reduce the error components that appear oscillatory on these grids. Finally, the problem can be prolongated back to the fine energy grid where the troublesome error modes have now been removed. 

%A multilevel method has never been applied to the energy portion of phase space for the neutron transport problem before. The smoother that will be used for the multilevel energy preconditioner will be either Jacobi or successive over relaxation. The new preconditioner will also take advantage of the energy set decomposition such that it can be parallelized. It will be tested on problems with many energy groups to provide a wide range of grid and set combinations. 

%The second type of preconditioner will be a splitting of the shifted scattering matrix and will be available for eigenvalue problems only. The shifted scattering matrix will be split into upper- and lower-triangular pieces. Then $(\ve{I} - \ve{T\tilde{S}}_L)^{-1} = \ve{M}^{-1}$ will be applied to the operator form of the transport equation as a traditionally-formulated left preconditioner. 

%The intention is that $\ve{M}^{-1}$ will be sufficiently close to $\ve{A}^{-1}$ to provide acceleration. It will be fairly inexpensive to form $\ve{M}^{-1}$. This preconditioner will also be parallelized. The shifted scattering preconditioner will be tested on criticality problems that are traditionally difficut for \Sn methods to solve. Because the shifted scattering matrix is new, the preconditioner is also  new.

%Both preconditioners will be used on a toy problem for verification. They will also be applied to a large scale real PWR problem to investigate their influence on problems of interest. One of the goals of this research is to categorize which types of problems benefit from which preconditioners. An important and related area of research will be the tradeoff between the cost of construction and application of the preconditioners, and the change in solution time as a function of parallelization. 

%\section{Conclusion}
%Denovo can now be decomposed in energy for fixed source calculations. Test problems showed the new method exhibited good strong and weak scaling in energy. This capability will soon be available for eigenvalue problems. Parallelizing over this part of phase space allows Denovo to overcome the limitations inherent in KBA. Thus, Denovo will be able to scale to $O(100,000)$ cores for all challenging problems. 

%The new solution methods for fixed source and eigenvalue problems not only enable parallelization in energy, but will also reduce time to solution for the same number of processors. Using a Krylov solver instead of Gauss Seidel for fixed source problems improved convergence for the problems tested. The addition of the shifted RQI method will similarly accelerate criticality problems with high dominance ratios. This improvement will be most beneficial for the problems that are currently the most difficult to converge.

%Finally, adding new preconditioners should reduce the number of iterations needed for a given calculation. While the benefit of the proposed preconditioners is the least certain, it is likely they will be useful for at least some problems. 

%Overall, the research proposed in this document will accelerate Denovo in multiple ways. This acceleration is crucial for enabling the solution of today's ``grand challenge'' problems. It is hoped that improved methods will lead to improved reactor designs and systems, and that the frontier of computational challenges will be moved forward.  


